\documentclass[letterpaper, 11pt]{article}
% \usepackage{fontspec}

% ==================================================

% document parameters
% \usepackage[spanish, mexico, es-lcroman]{babel}
\usepackage[english]{babel}
\usepackage[margin = 1in]{geometry}

% ==================================================

% Packages for math
\usepackage{mathrsfs}
\usepackage{amsfonts}
\usepackage{amsmath}
\usepackage{amsthm}
\usepackage{amssymb}
\usepackage{physics}
\usepackage{dsfont}
\usepackage{esint}
\usepackage{graphicx}
\usepackage{listings}

% ==================================================

% Packages for writing
\usepackage{enumerate}
\usepackage[shortlabels]{enumitem}
\usepackage{framed}
\usepackage{csquotes}

% ==================================================

% Miscellaneous packages
\usepackage{float}
\usepackage{tabularx}
\usepackage{xcolor}
\usepackage{multicol}
\usepackage{subcaption}
\usepackage{caption}
\captionsetup{format = hang, margin = 10pt, font = small, labelfont = bf}

% Citation
\usepackage[round, authoryear]{natbib}

% Hyperlinks setup
\usepackage{hyperref}
\definecolor{links}{rgb}{0.36,0.54,0.66}
\hypersetup{
   colorlinks = true,
    linkcolor = black,
     urlcolor = blue,
    citecolor = blue,
    filecolor = blue,
    pdfauthor = {Author},
     pdftitle = {Title},
   pdfsubject = {subject},
  pdfkeywords = {one, two},
  pdfproducer = {LaTeX},
   pdfcreator = {pdfLaTeX},
   }

% ==================================================

% Plotting
\usepackage{pgfplots}
% For better compatibility with pdfLaTeX
\pgfplotsset{compat=1.18}

\usepackage{titlesec}
\usepackage[many]{tcolorbox}

% Adjust spacing after the chapter title
\titlespacing*{\chapter}{0cm}{-2.0cm}{0.50cm}
\titlespacing*{\section}{0cm}{0.50cm}{0.25cm}

% Indent 
\setlength{\parindent}{0pt}
\setlength{\parskip}{1ex}

% --- Theorems, lemma, corollary, postulate, definition ---
% \numberwithin{equation}{section}

\newtcbtheorem[]{problem}{Problem}%
    {enhanced,
    colback = black!5, %white,
    colbacktitle = black!5,
    coltitle = black,
    boxrule = 0pt,
    frame hidden,
    borderline west = {0.5mm}{0.0mm}{black},
    fonttitle = \bfseries\sffamily,
    breakable,
    before skip = 3ex,
    after skip = 3ex
}{problem}

\tcbuselibrary{skins, breakable}

% --- You can define your own color box. Just copy the previous \newtcbtheorm definition and use the colors of yout liking and the title you want to use.
% --- Basic commands ---
%   Euler's constant
\newcommand{\eu}{\mathrm{e}}

%   Imaginary unit
\newcommand{\im}{\mathrm{i}}

%   Sexagesimal degree symbol
\newcommand{\grado}{\,^{\circ}}

%   Easy reals
\newcommand{\R}{\mathbb{R}}

% --- Linear Algebra ---
% Matrix transpose
\newcommand{\transpose}[1]{{#1}^{\mathsf{T}}}

%%% Calculus
%   Definite integral from -\infty to +\infty
\newcommand{\Int}{\int\limits_{-\infty}^{\infty}}

%   Indefinite integral
\newcommand{\rint}[2]{\int{#1}\dd{#2}}

%  Definite integral
\newcommand{\Rint}[4]{\int\limits_{#1}^{#2}{#3}\dd{#4}}

%   Dot product symbol (use the command \bigcdot)
\makeatletter
\newcommand*\bigcdot{\mathpalette\bigcdot@{.5}}
\newcommand*\bigcdot@[2]{\mathbin{\vcenter{\hbox{\scalebox{#2}{$\m@th#1\bullet$}}}}}
\makeatother

%   Hamiltonian
\newcommand{\Ham}{\hat{\mathcal{H}}}

%   Trace
\renewcommand{\Tr}{\mathrm{Tr}}

% Christoffel symbol of the second kind
\newcommand{\christoffelsecond}[4]{\dfrac{1}{2}g^{#3 #4}(\partial_{#1} g_{#2 #4} + \partial_{#2} g_{#1 #4} - \partial_{#4} g_{#1 #2})}

% Riemann curvature tensor
\newcommand{\riemanncurvature}[5]{\partial_{#3} \Gamma_{#4 #2}^{#1} - \partial_{#4} \Gamma_{#3 #2}^{#1} + \Gamma_{#3 #5}^{#1} \Gamma_{#4 #2}^{#5} - \Gamma_{#4 #5}^{#1} \Gamma_{#3 #2}^{#5}}

% Covariant Riemann curvature tensor
\newcommand{\covariantriemanncurvature}[5]{g_{#1 #5} R^{#5}{}_{#2 #3 #4}}

% Ricci tensor
\newcommand{\riccitensor}[5]{g_{#1 #5} R^{#5}{}_{#2 #3 #4}}


\begin{document}

\begin{Large}
    \textsf{\textbf{Homework - October 1, 2024}}
\end{Large}

\vspace{1ex}

\textsf{\textbf{Student:}} \text{Ayrton Chilibeck}, \href{mailto:achilibe@ualberta.ca}{\texttt{achilibe@ualberta.ca}}\\
\textsf{\textbf{Lecturer:}} \text{Lili Mou}, \href{mailto:UoA.F24.466566@gmail.com}{\texttt{UoA.F24.466566@gmail.com}}


\vspace{2ex}

\begin{problem}{The Epsilon-Neighbourhood}{e-neighbourhood}
Prove that, given
\begin{equation*}
  \lambda = 1-\frac{\epsilon}{2||y-x||}
\end{equation*}
and
\begin{equation*}
  z = \lambda x + (1-\lambda)y
\end{equation*}
That $z$ is in the $\epsilon$-neighbourhood of $x$.
\end{problem}

To prove Baye's Theorem, we recall the definition of conditional probablity:

\begin{align*}
  P(X|Y) =& \frac{P(X\cap Y)}{P(Y)}\\
  P(Y|X) =& \frac{P(X\cap Y)}{P(X)}
\end{align*}
Solving for $P(X\cap Y)$ in both equations, we can then substitute, and set the two systems equal to each other
\begin{equation*}
  P(X|Y)P(Y) = P(Y|X)P(Y) = P(X\cap Y)
\end{equation*}
Rearranging gives the default form of the equation
\begin{equation*}
  P(Y|X) = \frac{P(X|Y)P(Y)}{P(X)}
\end{equation*}
We can then use the multiplication rule for conditional probablilities to write
\begin{equation}
  P(Y|X) = \frac{P(X|Y)P(Y)}{\sum_{j}P(X_{j})P(Y|X_{j})}
\end{equation}
as required.\qed


\begin{problem}{Global optimum of a Convex Function}{global-optimum}
Prove that the global optimum of a convex function $f$ is where $\nabla f = 0$.
\end{problem}

Recall the definition of the L2 penalized MSE:
\[
  J(\vec{w}^{(t)}) = \frac{1}{2M} \sum_{i=1}^{M} \left(\\sum_{i=0}^d w_{i}x_{i}^{(m)} - t^{(m)}\right)^2 + \lambda \sum_{i=0}^{d}w_{i}^{2}
\]
we can rewrite this in matrix notation as follows:
\[
  J(\vec{w}^{(t)}) = \frac{1}{2M} (X^\top w - \vec{t})^{\top}(X^{\top}w-\vec{t}) + \lambda |\vec{w}_{i}|^{2}_{2}
\]
Our goal is to compute the optimal $\vec{w}$ for our algorithm, which is one such that we minimize the loss. If we take the gradient with respect to $\vec{w}$, we can solve for the optimal $\vec{w}$ as follows:
\begin{align*}
  \nabla_{w} J(\vec{w}^{(t)}) &= \frac{1}{M}X^{\top}(X\vec{w} - \vec{t}) + 2\lambda\vec{w}\\
  \vec{0}&= X^{\top}(X\vec{w} - \vec{t}) + M\lambda\vec{w} && \text{F.O. Condition}\\
  \vec{w}&=\left(X^{\top}X + M\lambda I\right)^{-1}(X^{\top}\vec{t})
\end{align*}
Which gives us the optimal $w$ for our algorithm.\qed


\begin{problem}{Learning-Rate Annealing Schedule}{lr-schedule}
 In the gradient descent algorithm, $\alpha >0$ is the learning rate. If  is small enough, then the function value guarantees to decrease. In practice, we may anneal $\alpha$, meaning that we start from a relatively large $\alpha$, but decrease it gradually.

Show that $\alpha$ cannot be decreased too fast. If $\alpha$ is decreased too fast, even if it is strictly positive, the gradient descent algorithm may not converge to the optimum of a convex function.
\end{problem}

Recall the general form for a gradient descent optimization algorithm:
\begin{enumerate}
  \item Initialize the weights
  \item Check the gradient at the starting point
  \item modify the weights according to a learning rate
  \item repeat starting at step 2 until the loss is below a certain threshold
\end{enumerate}
We need to know how to calculate the gradient of the L1 loss in order to follow the steps above, so we can compute it as follows:
\begin{align*}
  L(\vec{w}) &= \frac{1}{2M} \sum_{i=1}^{n} \left(\sum_{i=0}^{d}x_i w_{i}^{(m)} - t^{(m)}\right) + \lambda \sum_{i=0}^{d}|w_{i}| \\
  L(\vec{w}) &= \frac{1}{2M} \left(X\vec{w} - \vec{t}\right)^{\top}\left(X\vec{w} - \vec{t}\right) + \lambda |\vec{w}| \\
  \nabla_{w}L(\vec{w}) &= \frac{1}{M}X^{\top}(X\vec{w}-\vec{t})\footnote{Recall our discussion on proximal methods. We only need to care about the MSE term for our gradient descent}
\end{align*}

We also need to recall the proximal operator:
\[
\text{prox}(w, \tau) =
\begin{cases}
w - \tau & \text{if } w > \tau \\
0 & \text{if } |w| \leq \tau \\
w + \tau & \text{if } w < -\tau
\end{cases}
\]

With these results, we can provide the pseudocode for the gradient-based optimization of $\vec{w}$ using the L1 penalized MSE as shown in algorithm \ref{alg:l1gd}

\begin{algorithm}
    \KwIn{Initial weights \( \vec{w}^{(0)} \), learning rate \( \eta \), maximum iterations \( T \)}
    \For{$t = 0$ \KwTo $T-1$}{
        $\nabla_{w}L(\vec{w}) \gets \frac{1}{M}X^{\top}(X\vec{w}-\vec{t})$\;
        $\vec{w}^{(t+1)} \gets \vec{w}^{(t)} - \eta \nabla J(\vec{w}^{(t)})$\;
        $\vec{w}^{(t+1)} \gets \text{prox}(\vec{w}^{(t+1)}, \lambda \eta)$\;
    }
\caption{Gradient-Based Optimization Algorithm}\label{alg:l1gd}
\end{algorithm}


% =================================================

% \newpage

% \vfill

\bibliographystyle{apalike}
\bibliography{refs}

\end{document}
