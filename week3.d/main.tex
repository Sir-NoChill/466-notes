\input{preamble}
\input{format}
\input{commands}

\begin{document}

\begin{Large}
    \textsf{\textbf{Homework - October 1, 2024}}
\end{Large}

\vspace{1ex}

\textsf{\textbf{Student:}} \text{Ayrton Chilibeck}, \href{mailto:achilibe@ualberta.ca}{\texttt{achilibe@ualberta.ca}}\\
\textsf{\textbf{Lecturer:}} \text{Lili Mou}, \href{mailto:UoA.F24.466566@gmail.com}{\texttt{UoA.F24.466566@gmail.com}}


\vspace{2ex}

\begin{problem}{The Epsilon-Neighbourhood}{e-neighbourhood}
Prove that, given
\begin{equation*}
  \lambda = 1-\frac{\epsilon}{2||y-x||}
\end{equation*}
and
\begin{equation*}
  z = \lambda x + (1-\lambda)y
\end{equation*}
That $z$ is in the $\epsilon$-neighbourhood of $x$.
\end{problem}

To prove Baye's Theorem, we recall the definition of conditional probablity:

\begin{align*}
  P(X|Y) =& \frac{P(X\cap Y)}{P(Y)}\\
  P(Y|X) =& \frac{P(X\cap Y)}{P(X)}
\end{align*}
Solving for $P(X\cap Y)$ in both equations, we can then substitute, and set the two systems equal to each other
\begin{equation*}
  P(X|Y)P(Y) = P(Y|X)P(Y) = P(X\cap Y)
\end{equation*}
Rearranging gives the default form of the equation
\begin{equation*}
  P(Y|X) = \frac{P(X|Y)P(Y)}{P(X)}
\end{equation*}
We can then use the multiplication rule for conditional probablilities to write
\begin{equation}
  P(Y|X) = \frac{P(X|Y)P(Y)}{\sum_{j}P(X_{j})P(Y|X_{j})}
\end{equation}
as required.\qed


\begin{problem}{Global optimum of a Convex Function}{global-optimum}
Prove that the global optimum of a convex function $f$ is where $\nabla f = 0$.
\end{problem}

\begin{enumerate}
  \item Linear regression by Gradient Descent:
  \begin{enumerate}
    \item Best epoch: 99
    \item Best epoch validation risk: 0.36548514
    \item Test risk: 0.35231195
  \end{enumerate}
  \begin{figure}[h]
      \centering
      \begin{minipage}{0.49\textwidth}
          \centering
	  \includegraphics[scale=0.49]{../images/q2a_val_loss.png}
          \caption{Validation loss}
      \end{minipage}
      \hfill
      \begin{minipage}{0.49\textwidth}
          \centering
	  \includegraphics[scale=0.49]{../images/q2a_val_risk.png}
          \caption{Validation risk}
      \end{minipage}
  \end{figure}
  \item Linear regression with quadratic parameters, we found that the best values were
  \begin{enumerate}
    \item Best epoch: 99
    \item Best epoch validation risk: 0.3358987
    \item Test risk: 0.38372331
  \end{enumerate}
  \begin{figure}[h]
      \centering
      \begin{minipage}{0.49\textwidth}
          \centering
	  \includegraphics[scale=0.49]{../images/q2b_train_loss.png}
          \caption{Validation loss}
      \end{minipage}
      \hfill
      \begin{minipage}{0.49\textwidth}
          \centering
	  \includegraphics[scale=0.49]{../images/q2b_val_risk.png}
          \caption{Validation risk}
      \end{minipage}
  \end{figure}
  \item My research question would be does this method converge with the quadratic parameters, and does the test loss increase relative to the simple linear regression with more epochs. I found that the system continues to converge past the 100 epochs, changing up to 1000 epochs and has lower test accuracy than the linear regression. This means our quadratic augmentation overfits the data, so our simple linear regression is the more descriptive model.
\end{enumerate}


\begin{problem}{Learning-Rate Annealing Schedule}{lr-schedule}
 In the gradient descent algorithm, $\alpha >0$ is the learning rate. If  is small enough, then the function value guarantees to decrease. In practice, we may anneal $\alpha$, meaning that we start from a relatively large $\alpha$, but decrease it gradually.

Show that $\alpha$ cannot be decreased too fast. If $\alpha$ is decreased too fast, even if it is strictly positive, the gradient descent algorithm may not converge to the optimum of a convex function.
\end{problem}

Recall the general form for a gradient descent optimization algorithm:
\begin{enumerate}
  \item Initialize the weights
  \item Check the gradient at the starting point
  \item modify the weights according to a learning rate
  \item repeat starting at step 2 until the loss is below a certain threshold
\end{enumerate}
We need to know how to calculate the gradient of the L1 loss in order to follow the steps above, so we can compute it as follows:
\begin{align*}
  L(\vec{w}) &= \frac{1}{2M} \sum_{i=1}^{n} \left(\sum_{i=0}^{d}x_i w_{i}^{(m)} - t^{(m)}\right) + \lambda \sum_{i=0}^{d}|w_{i}| \\
  L(\vec{w}) &= \frac{1}{2M} \left(X\vec{w} - \vec{t}\right)^{\top}\left(X\vec{w} - \vec{t}\right) + \lambda |\vec{w}| \\
  \nabla_{w}L(\vec{w}) &= \frac{1}{M}X^{\top}(X\vec{w}-\vec{t})\footnote{Recall our discussion on proximal methods. We only need to care about the MSE term for our gradient descent}
\end{align*}

We also need to recall the proximal operator:
\[
\text{prox}(w, \tau) =
\begin{cases}
w - \tau & \text{if } w > \tau \\
0 & \text{if } |w| \leq \tau \\
w + \tau & \text{if } w < -\tau
\end{cases}
\]

With these results, we can provide the pseudocode for the gradient-based optimization of $\vec{w}$ using the L1 penalized MSE as shown in algorithm \ref{alg:l1gd}

\begin{algorithm}
    \KwIn{Initial weights \( \vec{w}^{(0)} \), learning rate \( \eta \), maximum iterations \( T \)}
    \For{$t = 0$ \KwTo $T-1$}{
        $\nabla_{w}L(\vec{w}) \gets \frac{1}{M}X^{\top}(X\vec{w}-\vec{t})$\;
        $\vec{w}^{(t+1)} \gets \vec{w}^{(t)} - \eta \nabla J(\vec{w}^{(t)})$\;
        $\vec{w}^{(t+1)} \gets \text{prox}(\vec{w}^{(t+1)}, \lambda \eta)$\;
    }
\caption{Gradient-Based Optimization Algorithm}\label{alg:l1gd}
\end{algorithm}


% =================================================

% \newpage

% \vfill

\bibliographystyle{apalike}
\bibliography{refs}

\end{document}
