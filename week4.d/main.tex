\documentclass[letterpaper, 11pt]{article}
% \usepackage{fontspec}

% ==================================================

% document parameters
% \usepackage[spanish, mexico, es-lcroman]{babel}
\usepackage[english]{babel}
\usepackage[margin = 1in]{geometry}

% ==================================================

% Packages for math
\usepackage{mathrsfs}
\usepackage{amsfonts}
\usepackage{amsmath}
\usepackage{amsthm}
\usepackage{amssymb}
\usepackage{physics}
\usepackage{dsfont}
\usepackage{esint}
\usepackage{graphicx}
\usepackage{listings}

% ==================================================

% Packages for writing
\usepackage{enumerate}
\usepackage[shortlabels]{enumitem}
\usepackage{framed}
\usepackage{csquotes}

% ==================================================

% Miscellaneous packages
\usepackage{float}
\usepackage{tabularx}
\usepackage{xcolor}
\usepackage{multicol}
\usepackage{subcaption}
\usepackage{caption}
\captionsetup{format = hang, margin = 10pt, font = small, labelfont = bf}

% Citation
\usepackage[round, authoryear]{natbib}

% Hyperlinks setup
\usepackage{hyperref}
\definecolor{links}{rgb}{0.36,0.54,0.66}
\hypersetup{
   colorlinks = true,
    linkcolor = black,
     urlcolor = blue,
    citecolor = blue,
    filecolor = blue,
    pdfauthor = {Author},
     pdftitle = {Title},
   pdfsubject = {subject},
  pdfkeywords = {one, two},
  pdfproducer = {LaTeX},
   pdfcreator = {pdfLaTeX},
   }

% ==================================================

% Plotting
\usepackage{pgfplots}
% For better compatibility with pdfLaTeX
\pgfplotsset{compat=1.18}

\usepackage{titlesec}
\usepackage[many]{tcolorbox}

% Adjust spacing after the chapter title
\titlespacing*{\chapter}{0cm}{-2.0cm}{0.50cm}
\titlespacing*{\section}{0cm}{0.50cm}{0.25cm}

% Indent 
\setlength{\parindent}{0pt}
\setlength{\parskip}{1ex}

% --- Theorems, lemma, corollary, postulate, definition ---
% \numberwithin{equation}{section}

\newtcbtheorem[]{problem}{Problem}%
    {enhanced,
    colback = black!5, %white,
    colbacktitle = black!5,
    coltitle = black,
    boxrule = 0pt,
    frame hidden,
    borderline west = {0.5mm}{0.0mm}{black},
    fonttitle = \bfseries\sffamily,
    breakable,
    before skip = 3ex,
    after skip = 3ex
}{problem}

\tcbuselibrary{skins, breakable}

% --- You can define your own color box. Just copy the previous \newtcbtheorm definition and use the colors of yout liking and the title you want to use.
% --- Basic commands ---
%   Euler's constant
\newcommand{\eu}{\mathrm{e}}

%   Imaginary unit
\newcommand{\im}{\mathrm{i}}

%   Sexagesimal degree symbol
\newcommand{\grado}{\,^{\circ}}

%   Easy reals
\newcommand{\R}{\mathbb{R}}

% --- Linear Algebra ---
% Matrix transpose
\newcommand{\transpose}[1]{{#1}^{\mathsf{T}}}

%%% Calculus
%   Definite integral from -\infty to +\infty
\newcommand{\Int}{\int\limits_{-\infty}^{\infty}}

%   Indefinite integral
\newcommand{\rint}[2]{\int{#1}\dd{#2}}

%  Definite integral
\newcommand{\Rint}[4]{\int\limits_{#1}^{#2}{#3}\dd{#4}}

%   Dot product symbol (use the command \bigcdot)
\makeatletter
\newcommand*\bigcdot{\mathpalette\bigcdot@{.5}}
\newcommand*\bigcdot@[2]{\mathbin{\vcenter{\hbox{\scalebox{#2}{$\m@th#1\bullet$}}}}}
\makeatother

%   Hamiltonian
\newcommand{\Ham}{\hat{\mathcal{H}}}

%   Trace
\renewcommand{\Tr}{\mathrm{Tr}}

% Christoffel symbol of the second kind
\newcommand{\christoffelsecond}[4]{\dfrac{1}{2}g^{#3 #4}(\partial_{#1} g_{#2 #4} + \partial_{#2} g_{#1 #4} - \partial_{#4} g_{#1 #2})}

% Riemann curvature tensor
\newcommand{\riemanncurvature}[5]{\partial_{#3} \Gamma_{#4 #2}^{#1} - \partial_{#4} \Gamma_{#3 #2}^{#1} + \Gamma_{#3 #5}^{#1} \Gamma_{#4 #2}^{#5} - \Gamma_{#4 #5}^{#1} \Gamma_{#3 #2}^{#5}}

% Covariant Riemann curvature tensor
\newcommand{\covariantriemanncurvature}[5]{g_{#1 #5} R^{#5}{}_{#2 #3 #4}}

% Ricci tensor
\newcommand{\riccitensor}[5]{g_{#1 #5} R^{#5}{}_{#2 #3 #4}}


\begin{document}

\begin{Large}
    \textsf{\textbf{Homework - October 1, 2024}}
\end{Large}

\vspace{1ex}

\textsf{\textbf{Student:}} \text{Ayrton Chilibeck}, \href{mailto:achilibe@ualberta.ca}{\texttt{achilibe@ualberta.ca}}\\
\textsf{\textbf{Lecturer:}} \text{Lili Mou}, \href{mailto:UoA.F24.466566@gmail.com}{\texttt{UoA.F24.466566@gmail.com}}


\vspace{2ex}

\begin{problem}{Baye's Theorem}{Bayes}
Prove Baye's Theorem:
\begin{equation*}
  P(Y|X) = \frac{P(X|Y)P(Y)}{\sum_{j}P(X_{j})P(Y|X_{j})}
\end{equation*}
\end{problem}

To prove Baye's Theorem, we recall the definition of conditional probablity:

\begin{align*}
  P(X|Y) =& \frac{P(X\cap Y)}{P(Y)}\\
  P(Y|X) =& \frac{P(X\cap Y)}{P(X)}
\end{align*}
Solving for $P(X\cap Y)$ in both equations, we can then substitute, and set the two systems equal to each other
\begin{equation*}
  P(X|Y)P(Y) = P(Y|X)P(Y) = P(X\cap Y)
\end{equation*}
Rearranging gives the default form of the equation
\begin{equation*}
  P(Y|X) = \frac{P(X|Y)P(Y)}{P(X)}
\end{equation*}
We can then use the multiplication rule for conditional probablilities to write
\begin{equation}
  P(Y|X) = \frac{P(X|Y)P(Y)}{\sum_{j}P(X_{j})P(Y|X_{j})}
\end{equation}
as required.\qed


\begin{problem}{Monty Hall Problem}{monty-hall}
Consider the Monty Hall game in a TV show. There are three closed doors, behind which are a car and two goats placed randomly.

\begin{enumerate}
  \item You are asked to open a door by the host. Say, you would like to open Door 1. What is the probability of getting a car?
  \item The host knows where the car is but he/she does not tell you. Instead, the host will open another door with a goat.
        \begin{enumerate}
          \item If the car is behind Door 2, the host can only open Door 3.
          \item If the car is behind Door 1, the host can open either Door 2 or Door 3. He/she will do it with equal probability.
        \end{enumerate}
        Say, the host has opened Door 3. What is the probability of having the car behind Door 1 now? What is the probability of having the car behind Door 2 now?
  \item If my goal is to get the car, should I change my first choice (i.e., open Door 1 or Door 2)?
\end{enumerate}
\end{problem}

Recall the definition of the L2 penalized MSE:
\[
  J(\vec{w}^{(t)}) = \frac{1}{2M} \sum_{i=1}^{M} \left(\\sum_{i=0}^d w_{i}x_{i}^{(m)} - t^{(m)}\right)^2 + \lambda \sum_{i=0}^{d}w_{i}^{2}
\]
we can rewrite this in matrix notation as follows:
\[
  J(\vec{w}^{(t)}) = \frac{1}{2M} (X^\top w - \vec{t})^{\top}(X^{\top}w-\vec{t}) + \lambda |\vec{w}_{i}|^{2}_{2}
\]
Our goal is to compute the optimal $\vec{w}$ for our algorithm, which is one such that we minimize the loss. If we take the gradient with respect to $\vec{w}$, we can solve for the optimal $\vec{w}$ as follows:
\begin{align*}
  \nabla_{w} J(\vec{w}^{(t)}) &= \frac{1}{M}X^{\top}(X\vec{w} - \vec{t}) + 2\lambda\vec{w}\\
  \vec{0}&= X^{\top}(X\vec{w} - \vec{t}) + M\lambda\vec{w} && \text{F.O. Condition}\\
  \vec{w}&=\left(X^{\top}X + M\lambda I\right)^{-1}(X^{\top}\vec{t})
\end{align*}
Which gives us the optimal $w$ for our algorithm.\qed


\begin{problem}{Linearity of Expectation}{expectation}
  Prove that the Expectation function is linear:
  \begin{equation*}
    \mathbb{E}[\alpha f(X) + \beta g(X)] = \alpha\mathbb{E}[f(X)] + \beta\mathbb{E}[g(X)]
  \end{equation*}
\end{problem}

Recall the general form for a gradient descent optimization algorithm:
\begin{enumerate}
  \item Initialize the weights
  \item Check the gradient at the starting point
  \item modify the weights according to a learning rate
  \item repeat starting at step 2 until the loss is below a certain threshold
\end{enumerate}
We need to know how to calculate the gradient of the L1 loss in order to follow the steps above, so we can compute it as follows:
\begin{align*}
  L(\vec{w}) &= \frac{1}{2M} \sum_{i=1}^{n} \left(\sum_{i=0}^{d}x_i w_{i}^{(m)} - t^{(m)}\right) + \lambda \sum_{i=0}^{d}|w_{i}| \\
  L(\vec{w}) &= \frac{1}{2M} \left(X\vec{w} - \vec{t}\right)^{\top}\left(X\vec{w} - \vec{t}\right) + \lambda |\vec{w}| \\
  \nabla_{w}L(\vec{w}) &= \frac{1}{M}X^{\top}(X\vec{w}-\vec{t})\footnote{Recall our discussion on proximal methods. We only need to care about the MSE term for our gradient descent}
\end{align*}

We also need to recall the proximal operator:
\[
\text{prox}(w, \tau) =
\begin{cases}
w - \tau & \text{if } w > \tau \\
0 & \text{if } |w| \leq \tau \\
w + \tau & \text{if } w < -\tau
\end{cases}
\]

With these results, we can provide the pseudocode for the gradient-based optimization of $\vec{w}$ using the L1 penalized MSE as shown in algorithm \ref{alg:l1gd}

\begin{algorithm}
    \KwIn{Initial weights \( \vec{w}^{(0)} \), learning rate \( \eta \), maximum iterations \( T \)}
    \For{$t = 0$ \KwTo $T-1$}{
        $\nabla_{w}L(\vec{w}) \gets \frac{1}{M}X^{\top}(X\vec{w}-\vec{t})$\;
        $\vec{w}^{(t+1)} \gets \vec{w}^{(t)} - \eta \nabla J(\vec{w}^{(t)})$\;
        $\vec{w}^{(t+1)} \gets \text{prox}(\vec{w}^{(t+1)}, \lambda \eta)$\;
    }
\caption{Gradient-Based Optimization Algorithm}\label{alg:l1gd}
\end{algorithm}


\begin{problem}{Parameter Estimation}
  Let $X~U[a,b]$ be a continuous random variable uniformly distributed in the interval $[a, b]$ , where $a$ and $b$ are unknown parameters.

We have a dataset $\{x^{(m)}\}^{M}_{m=1}$ , where each data sample is iid drawn from the above distribution, and we would like to estimate the parameters $a$ and $b$.

Give the likelihood of parameters.

Give the maximum likelihood estimation of parameters.
\end{problem}

Recall the definition of linear regression in a probabilistic context:
\begin{align*}
  y_{i}=X_{i}^{\top}\vec{w} + \epsilon_{i}
\end{align*}
Where $\epsilon ~ \mathcal{N}(0, \sigma^{2})$. This means that $X^{\top}w$ should provide the expected value of $f(y)$ with some error given by a normal distribution. Since $\epsilon$ follows the Gaussian normal distribution, we can write
\begin{equation*}
  P(y|X, w) = \Pi_{i=0}^{d} \mathcal{N}(X_{i}^{t}\vec{w},\sigma^{2})
\end{equation*}
We then take the logarithm of this to reduce the problem to a summation
\begin{align*}
  \log\left(P(y|X, w)\right) &= \sum_{i=0}^{d}\left[\log\left(\frac{1}{\sqrt{2\pi\sigma^{2}}}\right) - \frac{X_{i}^{\top}\vec{w} - t_{i}}{2\sigma^{2}}\right] \\
  &=\frac-{d}{2}\log\left(2\pi\sigma^{2}\right) - \sum_{i=0}^{d}\left[ \frac{X_{i}^{\top}\vec{w} - t_{i}}{2\sigma^{2}}\right] \\
  &=\frac-{d}{2}\log\left(2\pi\sigma^{2}\right) - \frac{1}{2\sigma^{2}}\sum_{i=0}^{d}\left(X_{i}^{\top}\vec{w} - t_{i}\right)^{2}
\end{align*}

In order to apply the L1 regularization to this problem, we can add a regularizing factor to the expression:
\begin{equation*}
  P(\vec{w})\alpha \exp\left(-\lambda ||\vec{w}||_{1}\right)
\end{equation*}

We can find $P(\vec{w} | X, \vec{y})$ using Bayes' theorem
\begin{equation*}
  P(\vec{w}|X, \vec{t}) \alpha P(\vec{t} | X,\vec{w})\cdot P(\vec{w})
\end{equation*}
and since we know the distribution of $P(w)$, we can follow the steps to find the logarithmic version of $P(\vec{w}|X,\vec{t})$, appending the multiplication of $P(w)$ to get
\begin{equation*}
  P(\vec{w}|X,\vec{t}) = -\frac{d}{2}\log\left(2\pi\sigma^{2}\right) - \frac{1}{2\sigma^{2}}\sum_{i=0}^{d}\left(X_{i}^{\top}\vec{w} - t_{i}\right) -\lambda ||\vec{w}||_{1}
\end{equation*}

If we minimize this function, then we get the same equation as the L1 loss:
\begin{align*}
  \hat{w} &= \text{argmin}_{w}
  P(\vec{w}|X,\vec{t}) = \text{argmin}_{w}\left[-\frac{d}{2}\log\left(2\pi\sigma^{2}\right) - \frac{1}{2\sigma^{2}}\sum_{i=0}^{d}\left(X_{i}^{\top}\vec{w} - t_{i}\right)^{2} -\lambda ||\vec{w}||_{1}\right] \\
        &=\text{argmin}_{w}\left[\frac{1}{2\sigma^{2}}\sum_{i=0}^{d}\left(X_{i}^{\top}\vec{w} - t_{i}\right)^{2} -\lambda ||\vec{w}||_{1}\right]
\end{align*}

Conceptually, this means that minimizing the loss function and maximizing the expectation function of the probability distribution are equivalent, so minimizing the loss function is an accurate way to obtain an optimal weight matrix $w$ for a model.\qed


\begin{problem}{MSE}{mse}
  Suppose $M$ samples, each of which $x^{(m)}~\mathcal{N}(\mu,1)$ is iid generated. Show that the estimate
\begin{equation*}
  \hat{\mu} = \frac{1}{M}\sum^{M}_{m=1}x^{(m)}
\end{equation*}
is an unbiased estimate of $\mu$.
\end{problem}

This is a duplicate of the proof we covered in class for the MSE being an unbiased estimate of the value $\mu$.

First, the sample mean for the set is
\begin{equation*}
  \bar{\mu}=\frac{1}{M}\sum_{m=1}^{M}x^{(m)}
\end{equation*}

Now, we need to prove that the mean is a fair estimator, so let us compute the expectation of the gaussian normal distribution as given:
\begin{align*}
  \mathbb{E}[\mathcal{N}(\mu, \sigma^{2})] &= \text{argmax}_{\theta}\Pi P(x|\mu, \sigma)\\
                                           &= \text{argmax}_{\theta} \log\left[\text{argmax}_{\theta}\Pi P(x|\mu, \sigma)\right]\\
  &=\text{argmax}_{\mu}\sum^{M}_{m=1}\log\left[\frac{1}{\sqrt{2\pi}\sigma}\exp\left\{-\frac{1}{2}\left(\frac{t^{(m)}-\mu^{T}x^{(m)}}{\sigma}\right)\right\}\right]\\
  \mathbb{E}[\mathcal{N}(\mu, \sigma^{2})] &= \frac{1}{M}\sum^{M}_{m=1}x^{(m)}
\end{align*}
Which we can find by a slight change to the derivation done in class.\qed

% =================================================

% \newpage

% \vfill

\bibliographystyle{apalike}
\bibliography{refs}

\end{document}
