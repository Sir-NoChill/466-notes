Recall the definition of concavity:
\begin{align*}
f(y)\geq f(x)+\nabla f(x)^{T}(y-x)
\end{align*}
If both $f$ and $g$ follow this definition, then we can write
\begin{equation*}
  f(y) + g(y) \geq f(x) + \nabla f(x)^{T}(y-x) + g(x) + \nabla g(x)^{T}(y-x)
\end{equation*}
I define $h = f + g$, so that we can write
\begin{equation*}
  h(y) \geq h(x) + \nabla f(x)^{T}(y-x) + \nabla g(x)^{T}(y-x)
\end{equation*}
Now, we can recall the definition of the gradient $\nabla$:
\begin{equation*}
\nabla f =
\begin{pmatrix}
\frac{\partial f}{\partial x_1} \\
\frac{\partial f}{\partial x_2} \\
\vdots \\
\frac{\partial f}{\partial x_n}
\end{pmatrix}
\end{equation*}
since we are multiplying a single term by this collumn matrix, I write
\begin{equation*}
  h(y) \geq h(x) + (\nabla f(x)^{T} + \nabla g(x)^{T})(y-x)
\end{equation*}
and we can further simplify to
\begin{equation*}
  h(y) \geq h(x) + (\nabla h(x)^{T})(y-x)
\end{equation*}
as required. We see that the sum of two convex functions will result in another convex function.\qed
